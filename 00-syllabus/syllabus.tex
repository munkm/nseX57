\documentclass[11pt, a4paper]{article}
\usepackage[inner=1in,outer=1in,top=1in,bottom=1in]{geometry}
\pagestyle{empty}
\usepackage{placeins}
\usepackage{graphicx}
\usepackage{fancyhdr, lastpage, bbding, pmboxdraw}
\usepackage{amsmath, amssymb}
\usepackage[usenames,dvipsnames]{color}
\definecolor{darkblue}{rgb}{0,0,.6}
\definecolor{darkred}{rgb}{.7,0,0}
\definecolor{darkgreen}{rgb}{0,.6,0}
\definecolor{red}{rgb}{.98,0,0}
\usepackage[colorlinks,pdfusetitle,urlcolor=darkblue,citecolor=darkblue,linkcolor=darkred,bookmarksnumbered,plainpages=false]{hyperref}
%\renewcommand{\thefootnote}{\fnsymbol{footnote}}

\pagestyle{fancyplain}
\fancyhf{}
\lhead{ \fancyplain{}{\CourseTitle} }
%\chead{ \fancyplain{}{} }
\rhead{ \fancyplain{}{\CourseSemester \CourseYear} }
%\rfoot{\fancyplain{}{page \thepage\ of \pageref{LastPage}}}
\fancyfoot[RO, LE] {page \thepage\ of \pageref{LastPage} }
\thispagestyle{plain}
\usepackage{tabularx}


%%%%%%%%%%%%%%%%%%%%%%%%%%%%%%%%%%%%
\usepackage{xspace}

\newcommand{\CourseNumber}{NSE 457/557}
\newcommand{\CourseTitle}{Nuclear Reactor Laboratory\xspace}%
\newcommand{\CourseInstructor}{Dr. Madicken Munk\xspace}%
\newcommand{\CourseSemester}{Spring\xspace}%
\newcommand{\CourseYear}{2025\xspace}%
\newcommand{\CourseDays}{M\xspace}%
\newcommand{\CourseStart}{8:00\xspace}%
\newcommand{\CourseEnd}{8:50\xspace}%
\newcommand{\CourseInstructorEmail}{madicken.munk@oregonstate.edu}
\newcommand{\CourseRoom}{104C\xspace}%
\newcommand{\CourseBuilding}{Radiation Center\xspace}%
\newcommand{\CourseZoom}{}%
\newcommand{\CourseUniversity}{Oregon State University\xspace}%
\newcommand{\TeachingAssistant}{Grant Hendrickson \xspace}%
\newcommand{\TAOfficeHourDays}{Date Unknown \xspace}%
\newcommand{\TAOfficeHourStart}{TBA\xspace}%
\newcommand{\TAOfficeHourEnd}{TBA\xspace}%
\newcommand{\TAOfficeHourPlace}{220 Merryfield Hall\xspace}
\newcommand{\MunkOfficeHourDays}{Mondays\xspace}%
\newcommand{\MunkOfficeHourStart}{9:00 a.m.\xspace}%
\newcommand{\MunkOfficeHourEnd}{9:50 a.m.\xspace}%
\newcommand{\MunkOfficeHourPlace}{OSU Radiation Center Library\xspace}
%\newcommand{\Course<++>}{<++>}
%\newcommand{\Course<++>}{<++>}
%%%%%%%%%%%%%%%%%%%%%%%%%%%%%%%%%%%%
\title{\CourseNumber: \CourseTitle\\}
\author{\CourseUniversity}
\date{\CourseSemester \CourseYear}
\begin{document}
\maketitle
%\setlength{\unitlength}{1in}
\renewcommand{\arraystretch}{1.5}
\begin{center}
\begin{table}[h]
\begin{tabularx}{\textwidth}{rXrX}
\hline
\textbf{Instructor:} & \CourseInstructor & \textbf{Email:} & \href{mailto:\CourseInstructorEmail}{\CourseInstructorEmail} \\
\textbf{Time:} & \CourseDays \CourseStart -- \CourseEnd & \textbf{Place:} & \CourseRoom \CourseBuilding \\
\textbf{TA:} & \TeachingAssistant & & \\
\hline
\end{tabularx}
\end{table}
\end{center}

\paragraph{Course Pages:}
\begin{enumerate}
        \item \url{https://canvas.oregonstate.edu/courses/}
        \item \url{https://www.gradescope.com/courses/}
        \item \url{https://github.com/munkm/nseX57}
\end{enumerate}

\paragraph{TA Office Hours:} \TeachingAssistant is the TA for the course and will hold
office hours in \TAOfficeHourPlace.
They will hold office hours \TAOfficeHourDays from
\TAOfficeHourStart to \TAOfficeHourEnd.
Grade disputes will not be addressed in TA office hours.

\paragraph{Office Hours:} Dr. Munk  will hold office hours on
\MunkOfficeHourDays from \MunkOfficeHourStart to \MunkOfficeHourEnd in
\MunkOfficeHourPlace. Supplemental office hours are by appointment only
and should be requested with at least 24 hours notice.
Before making an appointment, please try the following options:
\begin{itemize}
\item If your colleagues might be helpful, please post your questions in the
        canvas discussion forum provided for this purpose.
\item If the TAs might be helpful, please attend their office hours.
\item Email Dr. Munk. If possible, please phrase your question such that it
        can be answered `Yes' or `No'.  Questions which require substantial
        response should be asked during the lecture or office hours.
\end{itemize}

If none of the above are successful or appropriate, you may email me with a
selection of times of your availability and we can find a time that is mutually
agreeable.

\paragraph{Main References:}
There are no required texts for this course. However, Dr. Munk does reccommend the following text as supplementary knowledge for the course:
\begin{itemize}
\item Duderstadt and Hamilton \cite{duderstadt_nuclear_1976} to supplement the reactor physics concepts covered. 
\item For assistance in computational aspects of the course material, textbooks \cite{mcclarren_computational_2017} and \cite{scopatz_effective_2015} are available online and in the library as an ebook.
\end{itemize}


\bibliographystyle{unsrt}
\renewcommand{\refname}{\normalfont\selectfont\normalsize}\vspace{-1cm}
\bibliography{nse457}

\paragraph{Course Learning Outcomes:}

This course will equip students to:
\begin{itemize}
\item Process data taken in the laboratory
\item Apply nuclear reactor theories to predict and explain the behavior of the OSU TRIGA reactor. 
\item Write professional laboratory reports. 
\end{itemize}

In addition, students enrolled in this course will demonstrate the ability to:

\begin{itemize}
\item Develop a procedure for a laboratory execrise using the Oregon Sttate Triga Reactor (OSTR) to investigate a phenomenon of hteir choice.
\item Use simulation tools, such as OpenMC, to compare predicted reactor behavior with measured data. 
\end{itemize}

\paragraph{Prerequisites:}
\begin{itemize}
\item NSE 451
\item NSE 452
\end{itemize}

\paragraph{Grading Policy:} Grades will be assigned as a weighted sum of the following work:

\begin{table}[h]
\begin{tabularx}{\textwidth}{Xr}
\textbf{Work} & \textbf{Weight}\\
\hline
\textbf{Lab Participation} & (10\%) \\
\textbf{Lab Report 1: Modeling the Subcritical Pile} & (15\%) \\
\textbf{Lab Report 2: Subcritical Pile} & (20\%) \\
\textbf{Lab Report 3: Control Rod Calibration} & (20\%) \\
\textbf{Lab Report 4: Pulse} & (20\%) \\
\textbf{Lab Report 5: Neutron Radiography} & (15\%) \\
\hline
\textbf{Total} & (100\%) \\
\end{tabularx}
\end{table}

\paragraph{Important Dates:}
\begin{center} \begin{minipage}{3.8in}
\begin{flushleft}
Lab 1 Deadline      \dotfill Monday, April 14, 11:59pm  \\
Lab 2 Deadline      \dotfill Monday, April 28, 11:59pm  \\
Lab 3 deadline      \dotfill Monday, May 19, 11:59pm  \\
Lab 4 deadline      \dotfill Monday, June 02 11:59pm  \\
Lab 5 deadline      \dotfill Monday, June 09, 11:59pm  \\
\end{flushleft}
\end{minipage}
\end{center}

\paragraph{Lab Sesssions:}

This course will have multiple laboratory sessions that will be held
around OSU's radiation center. \textbf{Only attend lab sessions to which you are registered.} We are limited in capacity for how many people can be in the control room at a single time and the reactor has been staffed to accomodate the sizes of our labs on each day. 

We will meet for the lab sessions in the lobby of the Radiation Center at 14:00 and depart for the reactor bay at 14:05. Your belongings will be securely stored by Ms. Aguilera in the RC entry office. Be on time for the lab; if you are late you may miss the escort to the reactor bay and lose points for the lab session attendance that week. 

We will be performing labs in a licensed facility. To attend lab sessions in the reactor bay, radiation center staff have communicated the following requirements for this class:

\begin{itemize}
\item All participants must show government-issued photo ID upon check-in. No copies or pictures of IDs are accepted -- please bring the actual card or passport. 
\item Cell phones, cameras, bags, and backpacks must be left in the front office prior to the tour. 
\item \textbf{No weapons} of any kind are permitted in the reactor building. 
\item Wear laboratory appropriate attire. This includes clothing that covers your feet, legs, and torso (long pants, closed shoes, and a full length shirt). 
\end{itemize} 


\paragraph{Class Policies:}

\begin{itemize}
\item[] \textbf{Integrity:} This is an institution of higher
learning. You are expected to uphold the principles of academic 
integrity at our institution and to be honest and ethical with your academic work. 
Academic dishonesty such as plagiarism, cheating, assisting, tampering, and falsification will not be 
tolerated. 
For reference, please note the \href{https://studentlife.oregonstate.edu/studentconduct/academic-misconduct-students}{Student Guide for Academic Misconduct} 
on OSU's Dean of Students page. 

\item[] \textbf{Attendance:} Regular attendance is expected. Your attendance and participation in laboratory sessions is mandatory and will contribute to your grade in this course. Request approval for absence for extenuating circumstances prior to absence.

\item[] \textbf{Electronics:} Active participation is essential and expected.
        Accordingly, students must turn off all electronic devices (laptop,
        tablets, cellphones, etc.) during class. Exceptions may be granted for
        laptops and tablets if engaging in computational exercises or taking notes. In lab sessions, your backpack, phone, and laptops must be left in the front entry with Ms. Aguilera and 
will not be permitted in the reactor bay or control room. 
\item[] \textbf{Collaboration:} Collaboratively reviewing course materials and
  working fellow students can be enriching.  This is
  recommended.  However, unless otherwise instructed, lab written assignments are
  to be completed independently and materials submitted as technical reports should be
  the result of one's own independent work. Dr. Munk recommends working through
  the labs exercises independently and then checking work with peers. Explaining your
  process is a good exercise to retain course material.
\item[] \textbf{Late Work:} Late lab assignments will not be accepted. Please plan accordingly. 

\item[] \textbf{Make-up Work:} There will be no negotiation about late work
        except in the case of absence documented by an 
\href{https://studentlife.oregonstate.edu/emergency-notifications}{emergency notification} from the Dean of Students.
        Dean of Students. Please note that such a letter is appropriate for many
        types of conflicts. 

\item[] \textbf{Grade Disputes:} It is important that you understand and agree
        with the grade you receive on your reports. If you would like
        to dispute your score, you must send an explanation by email to Dr.
        Munk within one week of recieving the grade.
        \textbf{Do not expect us to regrade anything while in conversation with
        you} as that would not be fair to the other students in the class, whose
        homeworks were graded without them present.  If you request a regrade,
        be aware that the it is possible that your score will go down.
        Regrade requests should be based on an error on our part (e.g., adding
        up the points incorrectly) or what you suspect is a misunderstanding of
        your work (e.g., arriving at the correct answer using an unexpected
        technique). Regrade requests that argue with the rubric (e.g., ``this is
        wrong, but you took too many points off'') will be returned without
        consideration.
        \textbf{Your work should stand alone.} If an assignment is disorganized or
        ambiguous, and requires an extensive explanation to the grader, you
        will likely still lose points. The labs not only evaluate your
        understanding of the material - they also evaluate your ability to
        communicate that understanding clearly.
\end{itemize}

\paragraph{University-Wide Course Statements:}

\begin{itemize}
\item[] \textbf{Academic Calendar:} All students are subject to the 
registration and refund deadlines as stated in the 
\href{https://registrar.oregonstate.edu/osu-academic-calendar}{Academic Calendar}

\item[] \textbf{Statement Regarding Students with Disabilities:} 
Accommodations for students with disabilities are determined and approved 
by Disability Access Services (DAS). If you, as a student, believe you are 
eligible for accommodations but have not obtained approval please contact 
DAS immediately at 541-737-4098 or at \url{http://ds.oregonstate.edu/}. 
DAS notifies students and faculty members of approved academic accommodations 
and coordinates implementation of those accommodations. While not required, 
students and faculty members are encouraged to discuss details of the 
implementation of individual accommodations.

\item[] \textbf{Student Conduct Expectations Link:} 
\url{https://beav.es/codeofconduct}

\item[] \textbf{Student Bill of Rights:} 
OSU has 
\href{https://asosu.oregonstate.edu/advocacy/rights}{twelve established student rights}. 
They include due process in all 
university disciplinary processes, an equal opportunity to learn, and grading 
in accordance with the course syllabus. 

\item[] \textbf{Reach out for Success:} 
University students encounter setbacks from time to time. If you encounter
 difficulties and need assistance, it’s important to reach out. Consider 
discussing the situation with an instructor or academic advisor. Learn about 
resources that assist with wellness and academic success at
\url{https://oregonstate.edu/ReachOut}. 
If you are in immediate crisis, please contact the Crisis Text Line by texting 
OREGON to 741-741 or call the National Suicide Prevention Lifeline at 
1-800-273-TALK (8255). 
\end{itemize}

\paragraph{Accessibility:} I hope that this course will be inclusive and
accommodating for all learners. 
To request
particular accommodations, please contact me as soon as possible so that we can
work out any necessary arrangements.

\paragraph{Safety:}
Emergencies can happen anywhere and at any time, so it’s important that we take
a minute to prepare for a situation in which our safety could depend on our
ability to react quickly. Take a moment to learn the different ways to leave
this building. If there's ever a fire alarm or something like that, you’ll know
how to get out and you'll be able to help others get out. Next, figure out the
best place to go in case of severe weather - we'll need to go to a low-level in
the middle of the building, away from windows. And finally, if there's ever
someone trying to hurt us, our best option is to run out of the building. If we
cannot do that safely, we'll want to hide somewhere we can't be seen, and we'll
have to lock or barricade the door if possible and be as quiet as we can. 
If we can't run or hide, we'll fight back with whatever we can
get our hands on. 
Remember you can sign up for
emergency text messages at \url{https://emergency.oregonstate.edu/emergency-management/osu-alerts}. This
\href{https://emergency.oregonstate.edu/emergency-preparedness/emergency-procedures/run-hide-fight-info}{Run Hide Fight Info page}
discusses the OSU's Run-Hide-Fight strategy.


\paragraph{Other Resources:}
University students typically experience a wide range of stressors during their
time on campus. Accordingly, campus resources exist to help students manage
stress levels, mental health, physical health, and emergencies while navigating
this environment. I hope you will take advantage of these campus resources.

\begin{itemize}
\item \href{https://recsports.oregonstate.edu/}{Campus Recreational Sports}
\item \href{https://counseling.oregonstate.edu/}{Counselling and Psychological Services}
\item \href{https://studenthealth.oregonstate.edu/}{Student Health Services}
\item \href{https://studentlife.oregonstate.edu/bnc}{The Basic Needs Center}
\end{itemize}

\pagebreak
\FloatBarrier
\renewcommand{\arraystretch}{1}
\begin{table}[h]
\begin{center}
\begin{tabular}{lllcccccc}
\multicolumn{8}{c}{\textbf{Course Schedule:}\textit{ Note that this schedule is
subject to change.}}\\
&&&&&&&&\\
\textbf{Date} & \textbf{Week} & \textbf{Day} & \textbf{Unit} & \textbf{Lab} & \textbf{Lab}& \textbf{Lab} \\
              &  &  &  & (M/T/W)  & \textbf{Given} & \textbf{Due}\\ \hline
\hline
3/31 & 1 & M & Intro/Syllabus  & OSTR Tour & 1  &   \\
4/7 & 2 & M & reactor design methodology\/ & -- &  &  \\
 & & & modeling the subcritical pile & & & \\
4/14 & 3 & M & subcritical pile physics,& subcritical pile & 2 & 1\\
 & & & model validation & & & \\
4/21 & 4 & M & model biases and    & -- &  &  \\
 & & & control rod calibrations & & & \\
4/28 & 5 & M & No Class                & -- &  & 2 \\
5/5 & 6 & M & OSTR physics, burnup        & CR calibration & 3  &  \\
 & & & in research reactors & & & \\
5/12 & 7 & M & flux mapping, power calibration & -- &  &  \\
5/19 & 8 & M & pulsing operations and        & pulse & 4  & 3 \\
 & & & reactor kinetics & & & \\
5/26 & 9 & M & Holiday                    & NRF lab & 5 &  \\
6/2 & 10 & M & neutron radiography        & NRF lab &  & 4\\
6/9 & 11 & M & Finals Week                & --  &  & 5  \\
\end{tabular}
\end{center}
\end{table}
\FloatBarrier



%%%%%% THE END
\end{document}
